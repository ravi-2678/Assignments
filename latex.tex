%b192678 cse-c4
%Latex Assignment
%Knitting Patterns,chess Notation,chemistry,Electrial circuits and 
%Typesetting exams
\doumentclass{knittingpattern}
\usepackage[utf8]{inputenc}
\usepackage{xcolor}
\usepackage{circuitikz}
\usepackage{amsmath}
\usepackage{tikz}
\usepackage{amssymb}
\usetikzlibrary{shapes,arrows,positioning}
\definecolor{colour0}{HTML}{000000}
\definecolor{colour1}{HTML}{EDB668}
\definecolor{colour2}{HTML}{000000}
\definecolor{colour3}{HTML}{d49fce}
%\usepackage{textwidth=16cm}{geometry}
\usepackage{graphicx}
\usepackage[table]{xcolor}
\usepackage{tabularx}
\usepackage{wrapfig}
\usepackage{xskak}
\usepackage{chemfig}
\usepackage{modiagram}
%opening
\begin{document}
\title{\textbf{\huge{Latex Assignment} }}
\author{\large {L Ravi \\ ID: B192678} }
\date{}
\maketitle
%\tableofcontents

\section{Knitting Patterns}
\begin{paragraph} This class provides a very convenient way to introduce boxed diagrams. We are thus going to use our stock image a few more
times. Also, it has a few features to make knitting instructions more readable, however, we can adapt them to make prettier documents for our purposes as well.

\end{paragraph}

%GRAPH
\begin{figure}[h]
 \centering
 \includegraphics[scale=0.5]{graph.jpg}
 \end{figure}
 

 \begin{table}[h]
 \centering
 \setlength{\arrayrulewidth}{0.56mm}
 \begin{tabular}{|c|}
     \hline
\rowcolor{yellow!80}{We have a way of highlighting important text, or as was originally intended, important}\\ 
\rowcolor{yellow!80}instructions. Feel free to choose whatever background and border colour you like when \\
\rowcolor{yellow!80}you replicate these features, but try to replicate the dimensions as well as you can.\\
     \hline
 \end{tabular}
 \end{table}
 
 \begin{table}[h]
\centering
\rowcolors{2}{blue!35}{green!30}
\begin{tabularx}{0.94\textwidth} 
{| >{\raggedright\arraybackslash}X  >{\centering\arraybackslash}X|  }
     \hline
     \cellcolor{blue!35}\textbf{Course}  & \cellcolor{blue!35}\textbf{Credits} \\
     \hline
     Introduction to Computer Programming & 6 \\
     \hline
     Abstractions and Paradigms in Programming & 6 \\
     \hline
     Abstractions and Paradigms in Programming Lab & 3 \\
     \hline
     Data Structure and Algorithms & 6 \\
     \hline
     Software Systems Lab & 8 \\
     \hline
\end{tabularx}
\end{table}
\note{colour2}{colour3}{NOTE}{This is a note.The feature was introduced to typeset a sequence of knitting instructions .This first coloumn is for the instruction,the second for the number of stitche.But they ,it looks 
cool !}
\newpage
  \begin{wrapfigure}{l}{0.45\textwidth} 
 \includegraphics[scale=0.5]{graph.jpg}
 \end{wrapfigure}
Look at the adjoining graph. Yes, you’ve seen it before. This
time, it is side by side with a paragraph! And there’s a beau-
tiful box around it! By default, this will be a quarter of the
width of the page. If you follow the hint that is the title of
this section, you won’t have to type in cumbersome code to
fit in images. Also, have you noticed that the pages are much
wider? A lot of it will be clear when you read up on the knit-
tingpatterns class. It is already available with the MacTeX
distributions, and of course, online on Overleaf. If your distri-
bution does not offer it, download it from here and copy the .cls
file to the folder/directory your code is in. See the point where
stuff becomes exponentially harder to do without LATEX? We
daresay the rest of this assignment crosses that point. Good
luck!
\vspace{2cm}
\newline
\section{Chess Notation}
\begin{center}
\newchessgame
\mainline{1.e4 e5 2.f4 e5xf4 3.Bc4 Qh4 4.Kf1 b5 5.Bc4xb5 Nf6 6.Nf3 Qh6 7.d3 Nh5 8.Nh4 Qg5 9.Nf5 c6 10.g4 Nf6 11.Rg1 c6xb5 12.h4 Qg6 13.h5 Qg5 14.Qf3 Ng8 15.Bc1xf4 Qf6 16.Nc3 Bc5 17.Nd5 Qf6xb2 18.Bd6}\\
\chessboard
\end{center}
\begin{center}
\newchessgame
\mainline{1.e4 e5 2.f4 e5xf4 3.Bc4 Qh4 4.Kf1 b5 5.Bc4xb5 Nf6 6.Nf3 Qh6 7.d3 Nh5 8.Nh4 Qg5 9.Nf5 c6 10.g4 Nf6 11.Rg1 c6xb5 12.h4 Qg6 13.h5 Qg5 14.Qf3 Ng8 15.Bc1xf4 Qf6 16.Nc3 Bc5 17.Nd5 Qf6xb2 18.Bd6 Bc5xg1 19.e5 Qb2xa1 20.Ke2 Na6 21.Ng7 Kd8 22.Qf6}\\
\chessboard
\end{center}

\begin{center}
\newchessgame
\mainline{1.e4 e5 2.f4 e5xf4 3.Bc4 Qh4 4.Kf1 b5 5.Bc4xb5 Nf6 6.Nf3 Qh6 7.d3 Nh5 8.Nh4 Qg5 9.Nf5 c6 10.g4 Nf6 11.Rg1 c6xb5 12.h4 Qg6 13.h5 Qg5 14.Qf3 Ng8 15.Bc1xf4 Qf6 16.Nc3 Bc5 17.Nd5 Qf6xb2 18.Bd6 Bc5xg1 19.e5 Qb2xa1 20.Ke2 Na6 21.Ng7 Kd8 22.Qf6 Ng8xf6 23.Be7#}\\
\chessboard
\end{center}
\section{Chemistry}
\subsection{Chemical Formulae}
\vspace{1cm}
\chemfig{*6((-Cl)-=(*6(-N=-=(-HN-[0.7]*6((-[2])--(*6(--N*6((-[2]-[0.7]-[7.3]OH)--)))))-))-=-=)}

\vspace{0.5cm}
{This is the molecule hydroxychloroquine, that recently shot to fame as a proposed cure for COVID-19. Please draw it. This
is a helpful Overleaf tutorial to help you get started.}
%\end{center}
\vspace{1cm}

\subsection{Molecular Orbital Diagrams}
\begin{center}
\begin{MOdiagram}
 \atom{left}{
     2p = {;up,up,up}
  }
  \atom{right}{
     2p = {;pair,up,up}
  }
  \molecule{
      2pMO = {;pair,pair,pair,up}
  }
\end{MOdiagram}
\end{center}

\begin{center}
\vspace{1cm}
\textbf{N \hspace{1cm} NO \hspace{1cm} O}
\end{center}
\vspace{1cm}
\section{Electrical Circuits}
\vspace{0.6cm}
\begin{center}
\begin{circuitikz}[american voltages]
\draw
 (0,0) to [\bipole{capacitor},o- ,l=$C_1$] (2,0)
  to [R,*-, l_=$R_1$] (2,3)
 to [short,-*] (4,3)
 to [R,l_=$R_L$,i=$i_C$] (4,0.8)
 to [short,*-o] (8,0.8)
 (2,0) to [R,-*, l_=$R_2$] (2,-3)
 to [short,-o] (0,-3)
 (4,3) to [short,-o] (9,3)
 (2,0) to [short,-|,i=$i_B$] (3.5,0)
 (4,-0.5) to [short,i=$i_E$] (4,-1.5)
  to [R,l_=$R_E$] (4,-3)
  to [short,*-*] (2,-3)
  (4,-1.5) to [short] (6,-1.5)
  to [\bipole{capacitor},l_=$C_2$] (6,-3)
  to [short,*-*] (4,-3)
  (6,-3) to [short,-o] (8,-3)
  to [short,-o] (9,-3)
  (8,0.5) to [open, v^>=$V_{out}$] (8,-3)
  (9,3) to [open, v^>=$V_{CC}$] (9,-3)
(2.5,0) to [open, v^>=$V_B$] (2.5,-3)
 (0,0) to [open, v^>=$V_0{sin({\omega}t)}$] (0,-3)
 (6,-3) to node[ground]{} (6,-4)
 (4.5,-1.5) to [open, v^>=$V_E$] (4.5,-3)
 (4.5,0.8)to [open, v^>=$V_{CE}$] (4.5,-1)
 (4,0.8) to [short,l_=$C$]  (4,0.5)
 (4,-1.0) to [short,l=$E$] (4,-0.5)
 (3.5,-0.3) to [short,l=$B$] (3.5,0.3)
 (3.5,0.1)to [short] (4,0.5)
 (3.5,-0.1)to [short,i=$ $] (4,-0.5)
\end{circuitikz}
\end{center}
\newpage
\section{Typesetting Exams}
\huge Maths \hspace{4cm} Assignment \hspace{3.5cm} IITB \# 
 \hrule 
\vspace{1cm}
  \textbf{\Large problem 1.}\Large{Show that there exists no non trivial unramified extension of \huge Q} \\
 
 \textbf{solution : }
     \Large{If K/Q is a non-trivial number field then $|disk \hspace{0.5cm} k|$ $>$. But then disk k has a prime factor so that so same prime ramifies in \huge k\\
   \hspace{10.5cm}}
     \\ \\
     \textbf{Problem 2.}  complete the following :\\ \\
     (a) \hspace{1cm} how does one prove a cot theorem ?\\ \\
     (b) \hspace{1cm} compute $\int cosx dx$\\ \\ 
     (c) \hspace{1cm} how does one square 
     $
\begin{pmatrix}
      a & b \\
      c & d 
\end{pmatrix} $? \\ \\
 \textbf{solutions : }\\ \\
  (a) use rollaries\\ \\
  (b) we have \\ \\
  \begin{equation}
   \hspace{7cm}
   \int cosx dx =sinx +c  
 \end{equation}
\hspace{1cm} we can check (1) 
  \hspace{2cm} $\frac{d}{dx} (sinx + c) = cosx$ \\ \\
 (c) This is routin. \hspace{12.5cm}$\square $ \\ \\ \\
\textbf{Problem 3.} Prove that $ \sqrt{2}$ is irrational. \\ \\
 \textbf{Proof :} . Asuume that $\sqrt{2}$=$\frac{a}{b}$, Where a,b  $\epsilon $  Z . Without loss of generality , we may assume gcd $(a,b)$=1 . Then we have \\
 
  \hspace{5cm}  $\sqrt{2}$=$\frac{a}{b}$ 
  
 \begin{equation}
     \sqrt{2}^2=\binom{a}{b}^2
 \end{equation}
 
\hspace{6.5cm} 2= $\frac{a^2}{b^2}$
 \begin{equation}
   a^2=2b^2
 \end{equation}
 But then from $(3)$,we know that $a^2$ is even so that a is even . But then we must have .
  \begin{center}
       $2a^2=b^2$
  \end{center}
so that $b^2$ is even , implying b is even.But then         gcd$(a,b)\ge 2 $ , a contradicion . \hspace{12.5cm}$\square $
 \\ \\ \\ \\
 (b)
 
    \includegraphics[width=7cm]{b.jpg}

\\ \\ 
\Large \textbf{4.Solving Puzzles \# 1} IN clas  we did three puzzles, the first one which is equivalent to finiteautomata. In general , a puzzle of this type has a frame like (but possibly with more/fewer squares and different colors) :
\begin{figure}[h]
\centering
 \includegraphics[width=10cm]{p1.jpg}
\end{figure} \\ 
 and a finite set of tiles like this (but possibly with more/fewer tiles and different colors):
 \begin{figure}[h]
     \centering
     \includegraphics[width=10cm]{p2.jpg}
 \end{figure}
 The tiles must be arranged so that adjastment areas have matching colors.there is an unlimited number of copies of each tile \\ \\
 (a) show how every puzzle of this type can be converted in to a  finite automation M and a string w that M accepts w if and only if the puzzle has a solution. \\ \\
 (b) Apply your construction to the above instance.\\ \\
 (c) Briefly describe how this gives an o(n) algorithm for solving puzzles of this type

\end{document}
